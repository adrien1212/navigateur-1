\documentclass[10pt,a4paper]{article}
\usepackage[french]{babel}
\usepackage[utf8]{inputenc}
\usepackage[T1]{fontenc}
\usepackage{times}
\usepackage{multicol}
\usepackage{geometry}
 \geometry{
 a4paper,
 total={170mm,257mm},
 left=20mm,
 top=20mm,
 }

\usepackage{tikz}

\usepackage{fancyvrb}



\usepackage{import}

\usepackage{hyperref}

\usepackage{listings}
\usepackage{color}

\definecolor{dkgreen}{rgb}{0,0.6,0}
\definecolor{gray}{rgb}{0.5,0.5,0.5}
\definecolor{mauve}{rgb}{0.58,0,0.82}
\definecolor{pgrey}{rgb}{0.46,0.45,0.48}
\definecolor{gray_frame}{RGB}{167,169,172}

\lstset{frame=bt,
  language=Java,
  aboveskip=3mm,
  belowskip=3mm,
  framerule=1pt,
  rulecolor=\color{gray_frame},
  showstringspaces=false,
  columns=flexible,
  basicstyle=\fontsize{10.5}{13}\selectfont\ttfamily,
  numbers=none,
  numberstyle=\tiny\color{gray},
  keywordstyle=\textbf,
  commentstyle=\textit,
  breaklines=true,
  breakatwhitespace=true,
  tabsize=4, 
  literate=
  {é}{{\'e}}1
  {è}{{\`{e}}}1
  {ê}{{\^{e}}}1
  {É}{{\'{E}}}1
  {Ê}{{\^{E}}}1
  {û}{{\^{u}}}1
  {ù}{{\`{u}}}1
  {â}{{\^{a}}}1
  {à}{{\`{a}}}1
  {á}{{\'{a}}}1
  {ã}{{\~{a}}}1
  {Á}{{\'{A}}}1
  {Â}{{\^{A}}}1
  {Ã}{{\~{A}}}1
  {ç}{{\c{c}}}1
  {Ç}{{\c{C}}}1
  {õ}{{\~{o}}}1
  {ó}{{\'{o}}}1
  {ô}{{\^{o}}}1
  {Õ}{{\~{O}}}1
  {Ó}{{\'{O}}}1
  {Ô}{{\^{O}}}1
  {î}{{\^{i}}}1
  {Î}{{\^{I}}}1
  {í}{{\'{i}}}1
  {Í}{{\~{Í}}}1, 
}

\usepackage{lipsum}
\usepackage{mwe}

\usepackage{floatrow}


\usepackage[simplified]{pgf-umlcd}
\renewcommand{\umlfillcolor}{white}
\renewcommand{\umldrawcolor}{black}


\usepackage{pgf-umlsd}

\usepackage{lscape}

\setlength{\parindent}{0pt} 
 
\usepackage{fancyhdr}
\usepackage{multirow}




\begin{document}
\thispagestyle{empty}
\newcommand{\ts}{\textsuperscript}

	\begin{center}
		\vspace*{3cm}
		
		%\textcolor{reddark} {{\fontsize{50}{60}\selectfont Rapport}} \\%
		{{\fontsize{50}{60}\selectfont Rapport}} \\
		\vspace*{2cm}
		
		\rule{1\textwidth}{1pt} \\
		\vspace*{0.5cm}
		\textbf{\huge{Technologie Objet \& Gestion de Projet}} \\
		\vspace*{0.3cm}
		\textbf{\huge{Navigateur Web}} \\
		\rule{0.75\textwidth}{2pt}
		
		\vspace*{2cm}
		
		\Large{Groupe 5 \\}
		\vspace*{0.3cm}
		\Large{Adrien CAUBEL \\ Guillaume CHAUDON\\ Simon JANDA \\ Simon MOULIN \\ Victor NANCHE } \\
		\vspace*{2cm}
		\large{1\ts{ère} année par apprentissage} \\
		\large{Promotion 2023}
		
		\vspace*{\fill}
		
		\includegraphics[scale=0.2]{inp-enseeiht.jpg}
	\end{center}


\thispagestyle{empty}
\tableofcontents

\newpage
\setcounter{page}{1}
\section{Introduction}
\subsection{Le projet}
Ce projet s'inscrit dans le cadre des unités d'enseignement Technologie Objet et Gestion de Projet dispensées durant le second semestre du cycle ingénieur. L'objectif principal de ce projet est de mettre en application les principes agiles vus en Gestion de Projet. L'implémentation de l'application quand à elle, est un objectif secondaire car les notions de programmation ont été travaillées en TDs et TPs. Il était donc conseillé d'avoir un projet ambitieux qui serait difficile à concrétiser dans son intégralité dans le temps imparti.\\

Après avoir assemblé notre équipe, nous nous sommes mis d’accord sur le projet que nous réaliserions : la création d'un navigateur internet. Par la suite, nous avons jugé important de consulter chaque membre de l’équipe afin de connaître sa vision et ses ambitions pour application. 

\subsection{Liens utiles}
L'avancement et les sources du projet sont disponibles aux liens suivants :
\begin{itemize}
\item Github \href{https://github.com/simjnd/navigateur/wiki}{https://github.com/simjnd/navigateur/wiki}
\item Trello \href{https://trello.com/b/HyinoLEE/navigateur}{https://trello.com/b/HyinoLEE/navigateur} 
\item SVN \href{http://cregut.svn.enseeiht.fr/2020/1air/cpo/pl/E5}{http://cregut.svn.enseeiht.fr/2020/1air/cpo/pl/E5}
\end{itemize}

\newpage

\section{Orginisation de l'équipe}
\subsection{Itération initiale}
Cette première itération a été consacré à l'organisation du travail. Ainsi, nous nous sommes mis d'accord sur les outils à utiliser. Nous avons donc créé un \href{https://trello.com/b/HyinoLEE/navigateur}{\underline{tableau Trello}} pour maintenir et mettre à jour le backlog  produits, les sprints et identifier le travail à réaliser, le travail en cours, les tâches terminées (Kanban). \\ 
Nous avons également déterminé les outils de travail, à savoir le framework JavaFX, l'outil de gestion Maven, les IDE IntelliJ/Eclipse et l'outil de contrôle des sources Git. \\ 
Enfin, sur un \href{https://github.com/simjnd/navigateur/wiki}{\underline{Wiki}} associé au depôt \href{https://github.com/simjnd/navigateur}{\underline{GitHub}} nous avons défini un workflow Git (conventions de gestion et de nommage des branches, du nombre de revues de validation des pull requests).

\subsection{Organisation générale}

Après avoir identifié les différentes épiques et user stories, nous avons procédé à un planning poker pour déterminer les points d'efforts et valeurs métier de ces stories. \\
Nous avons ensuite fournit le backlog du premier sprint, au début duquel nous avons chacun choisi les stories sur lesquelles nous souhaitions travailler. Au début des périodes de travail nous partagions des croissants pour du teambuilding et nous faisions un "daily" au cours duquel nous faisions rapport sur notre progrès, puis au cours des périodes de travail nous communiquions pour être sûr qu'aucun de nous ne soit bloqué.

\subsection{Organisation à chaque sprint}
Pour chaque sprint, nous reproduisions les étapes suivantes :
\begin{itemize}
\item Se donner un objectif pour le sprint.
\item Réalisation avec l'équipe des stories qui seront inluses dans le backlog du sprint.
\item Raffinage des points d'efforts et des valeurs métier pour chaque users stories.
\item Choix des users stories par le développeur.
\item \textit{Réalisation du backlog du sprint}
\item Sprint review : création d'une branche release, tag du commit, et démonstration.
\item Retrospective du sprint (à partir du deuxième) sur ce qu'on a apprécié et moins.
\end{itemize}

\newpage

\section{Découpage des fonctionnalités}
Avant de débuter la conception du projet, nous avons du établir les principales fonctionnalités et définir leur point d'effort et leur valeur métier. Une fois ces deux métriques définies nous avons établie les fonctionnalités à faire dans chaque sprint. 

\subsection{Liste des fonctionnalités}
\begin{table}[H]
\begin{tabular}{|c|l|l|l|}
\hline
\textbf{Epic}                           & \multicolumn{1}{c|}{\textbf{Fonctionnalité}} & \multicolumn{1}{c|}{\textbf{Point d'effort}} & \multicolumn{1}{c|}{\textbf{Valeur métier}} \\ \hline
\multirow{6}{*}{Fenetre}                & Avoir l'affichage                            & M                                            & 10                                          \\ \cline{2-4} 
                                        & Bouton rafraichir                            & S                                            & 8                                           \\ \cline{2-4} 
                                        & Bouton précédent                             & S                                            & 8                                           \\ \cline{2-4} 
                                        & Bouton suivant                               & S                                            & 8                                           \\ \cline{2-4} 
                                        & Barre de recherche                           & S                                            & 10                                          \\ \cline{2-4} 
                                        & Bouton Menu                                  & S                                            & 6                                           \\ \hline
\multirow{5}{*}{Historique}             & Enregistrer historique                       & XL                                           & 4                                           \\ \cline{2-4} 
                                        & Afficher historique                          & L                                            & 4                                           \\ \cline{2-4} 
                                        & Supprimer historique                         & S                                            & 4                                           \\ \cline{2-4} 
                                        & Rechercher historique                        & S                                            & 3                                           \\ \cline{2-4} 
                                        & Cliquer sur lien dans historique             & M                                            & 4                                           \\ \hline
\multirow{5}{*}{Onglet}                 & Ajouter onglet                               & XL                                           & 8                                           \\ \cline{2-4} 
                                        & Naviguer entre onglet                        & L                                            & 8                                           \\ \cline{2-4} 
                                        & Supprimer onglet                             & M                                            & 8                                           \\ \cline{2-4} 
                                        & Interchanger onglet                          & XL                                           & 3                                           \\ \cline{2-4} 
                                        & Grouper onglet                               & XL                                           & 3                                           \\ \hline
\multirow{5}{*}{Barre recherche}        & Ecrire dans la barre                         & M                                            & 5                                           \\ \cline{2-4} 
                                        & Loader (barre de chargement)                 & S                                            & 3                                           \\ \cline{2-4} 
                                        & Sauvegarder les cookies                      & L                                            & 6                                           \\ \cline{2-4} 
                                        & Suggestion basée sur historique              & L                                            & 6                                           \\ \cline{2-4} 
                                        & Choisir son moteur de recherche              & L                                            & 6                                           \\ \hline
\multirow{5}{*}{Favoris}                & Ajouter la page actuelle favoris             & M                                            & 7                                           \\ \cline{2-4} 
                                        & Modifier nom d'un favori                         & L                                            & 5                                           \\ \cline{2-4} 
                                        & Modifier adresse principale favori                     & L                                            & 5                                           \\ \cline{2-4} 
                                        & Créer un dossier                             & L                                            & 7                                           \\ \cline{2-4}
                                        & Cliquer sur un favori                             & S                                            & 7                                           \\ \cline{2-4} 
                                        & Modifier nom dossier                         & L                                            & 5                                           \\ \hline
\multirow{4}{*}{Mot de passe}           & Proposer génération lors d'une inscription   & L                                            & 4                                           \\ \cline{2-4} 
                                        & Saisie automatique du mot de passe           & XL                                           & 4                                           \\ \cline{2-4} 
                                        & Afficher liste des mots de passes            & M                                            & 4                                           \\ \cline{2-4} 
                                        & Détection des mots de passe pwned            & M                                            & 2                                           \\ \hline
\multirow{3}{*}{Téléchargement}         & Choix du dossier destination                 & M                                            & 4                                           \\ \cline{2-4} 
                                        & Etat du téléchargement                       & XL                                           & 5                                           \\ \cline{2-4} 
                                        & Liste des téléchargement                     & M                                            & 4                                           \\ \hline
\multirow{3}{*}{Données navigations}    & Sauvegarder cookies                          & L                                          & 6                                           \\ \cline{2-4} 
                                        & Sauvegarde localstorage                      & L                                           & 7                                         
                                             \\ \hline
\multicolumn{1}{|l|}{Navigation privée} & Navigation privée                            & M                                            & 6    
                                             \\ \hline

\end{tabular}
\end{table}


\let\thefootnote\relax\footnotetext{
Vous pouvez retrouver ces mêmes users story sur le \href{https://trello.com/b/HyinoLEE/navigateur}{\underline{Trello}}
}



\newpage
\subsection{Fonctionnalités sprint n°1}
Ce premier sprint a été consacré à la mise en place de l'infrastructure nécessaire pour l'utilisation d'un navigateur web. 
Cependant, étant donné que nous ne connaissions pas nos capacités, lors du \textit{sanity check} nous avons du ajouter de nouvelles tâches.
On retrouve ainsi les fonctionnalités suivantes.

\begin{table}[h]
\begin{tabular}{|c|l|l|l|l|}
\hline
\textbf{Epic}               & \multicolumn{1}{c|}{\textbf{Fonctionnalité}} & \multicolumn{1}{c|}{\textbf{Point d'effort}} & \multicolumn{1}{c|}{\textbf{Valeur métier}} & \textbf{Affectée à} \\ \hline
\multirow{6}{*}{Fenetre}    & Avoir l'affichage                            & M                                            & 10                                          & Victor              \\ \cline{2-5} 
                            & Bouton rafraichir                            & S                                            & 8                                           & Guillaume           \\ \cline{2-5} 
                            & Bouton précédent                             & S                                            & 8                                           & Guillaume           \\ \cline{2-5} 
                            & Bouton suivant                               & S                                            & 8                                           & Guillaume           \\ \cline{2-5} 
                            & Barre de recherche                           & S                                            & 10                                          & Victor              \\ \cline{2-5} 
                            & Bouton Menu                                  & S                                            & 6                                           & Simon J             \\ \hline
\multirow{2}{*}{Historique} & Enregistrer historique                       & XL                                           & 4                                           & Simon M, Adrien     \\ \cline{2-5} 
                            & Afficher historique                          & L                                            & 4                                           & Simon M, Adrien     \\ \hline
Barre recherche             & Loader (barre de chargement)                 & M                                            & 5                                           & Simon J             \\ \hline
\end{tabular}
\end{table}

Ainsi, nous avons pu calculer la vélocité de l'équipe. Étant donné l'utilisation de taille de T-shirt, nous avons défini une conversion arbitraire $S = 3, M = 5, L = 8, XL = 13$. Ainsi, la vélocité du premier sprint est de 46.

\subsubsection{Conclusion du sprint}
Ce premier sprint nous a permis d'avoir une première estimation de la vélocité de l'équipe. De plus, il y a eu une très bonne entente entre les membres de l'équipe. et toutes les tâches fixées ont été réalisées. Cela est donc prometteur pour la suite du projet.


\subsection{Fonctionnalités sprint n°2}
Deuxième sprint du projet, on se consacra plus sur la mise en place d'outils pour faciliter l'utilisation du navigateur.

\begin{table}[h]
\begin{tabular}{|c|l|l|l|l|}
\hline
\textbf{Epic}                            & \multicolumn{1}{c|}{\textbf{Fonctionnalité}} & \multicolumn{1}{c|}{\textbf{Point d'effort}} & \multicolumn{1}{c|}{\textbf{Valeur métier}} & \textbf{Affecté à} \\ \hline
\multirow{3}{*}{Onglet}                  & Ajouter un onglet                            & XL                                           & 8                                           & Adrien, Guillaume  \\ \cline{2-5} 
                                         & Naviguer entre onglet                        & L                                            & 8                                           & Adrien, Guillaume  \\ \cline{2-5} 
                                         & Supprimer un onglet                          & L                                            & 8                                           & Adrien, Guillaume  \\ \hline
\multirow{2}{*}{Historique}              & Rechercher historique                        & S                                            & 3                                           & Simon M, Victor    \\ \cline{2-5} 
                                         & Supprimer historique                         & S                                            & 4                                           & Simon M, Victor    \\ \hline
\multicolumn{1}{|l|}{Données navigation} & Sauvegarder cookies à la fermeture           & L                                            & 6                                           & Simon J            \\ \hline
\end{tabular}
\end{table}

La vélovieté théorique du sprint est de 48. Mais comme expliqué ci-dessous dans la conclusion du sprint, nous n'avons pas réussir à finir toutes les fonctionnalités.

\subsubsection{Conclusion du sprint}
Nous avons mal évalué la difficulté pour la mise en place des onglets de navigation. En effet, la tâche s'est avérée plus longue que prévue et de nombreux problèmes auxquels nous n'avions pas pensés nous ont obligé à modifier notre conception et une partie du code réalisé au premier sprint. \\
Par conséquent, l'épic \textit{onglet} sera continuée sur le troisième sprint.

\newpage

\subsection{Fonctionnalités sprint n°3}
Un léger retard pris durant le sprint précédent, nous devons donc terminer les tâches en suspends. Ensuite, nous implémenterons d'autres fonctionnalités utiles pour le confort utilisateur.

\begin{table}[h]
\begin{tabular}{|c|l|l|l|l|}
\hline
\textbf{Epic}                               & \multicolumn{1}{c|}{\textbf{Fonctionnalité}}                                                & \multicolumn{1}{c|}{\textbf{Point d'effort}} & \multicolumn{1}{c|}{\textbf{Valeur métier}} & \textbf{Affecté à} \\ \hline
\multirow{3}{*}{Onglet}                     & Ajouter un onglet                                                                           & XL                                           & 8                                           & Guillaume, Adrien  \\ \cline{2-5} 
                                            & Naviguer entre onglet                                                                       & L                                            & 8                                           & Guillaume, Simon J          \\ \cline{2-5} 
                                            & Supprimer un onglet                                                                         & L                                            & 8                                           & Guillaume, Simon J          \\ \hline
\multirow{3}{*}{Favoris}                    & Créer un dossier                                                                            & L                                            & 7                                           & Simon M            \\ \cline{2-5} 
                                            & Ajouter un favori                                                                           & M                                            & 7                                           & Simon M            \\ \cline{2-5} 
                                            & Cliquer sur un favori                                                                       & S                                            & 7                                           & Simon M             \\ \hline
Barre de recherche                          & \begin{tabular}[c]{@{}l@{}}Choisir son navigateur\\ (fichier de configuration)\end{tabular} & L                                            & 6                                           & Victor             \\ \hline
\multicolumn{1}{|l|}{Navigation privée}     & Navigation privée                                                                           & M                                            & 6                                           & Guillaume          \\ \hline
\multicolumn{1}{|l|}{Données de navigation} & Sauvegarde localstorage                                                                     & L                                            & 7                                           & Simon J            \\ \hline
\multirow{2}{*}{Autre}                      & Revue de code BDD                                                                           &                                              &                                             & Adrien             \\ \cline{2-5} 
                                            & Revue de code fichier de conf.                                                              &                                              &                                             & Adrien             \\ \hline
\end{tabular}
\end{table}

La fonctionnalité \textit{Choisir son navigateur} a également été sous-évaluée. Malgré qu'elle soit finie, elle nous a prise plus de temps que prévue. \\

La vélocité de ce sprint est de 69, ce qui est largement au-dessus des deux dernières vélocités. Mais, il faut considéré que l'épic \textit{onglet} s'est réalisée sur deux sprints. Mais, on peut quand même se demander si certain point d'effort n'ont pas été sur-estimés.

\subsubsection{Conclusion du sprint}
Ce troisème et dernier sprint nous a permis d'avoir une application réellement fonctionnelle.

\subsection{Fonctionnalités manquantes}
Ce dernier tableau résume les fonctionnalités qui n'ont pas pu être développées par manque de temps. Cependant, ces fonctionnalités sont les moins valorisantes pour l'application, là, où, nous avons voulu nous concentrer sur les services essentiels.

\begin{table}[h]
\begin{tabular}{|c|l|l|l|}
\hline
\textbf{Epic}            & \multicolumn{1}{c|}{\textbf{Fonctionnalité}} & \multicolumn{1}{c|}{\textbf{Point d'effort}} & \multicolumn{1}{c|}{\textbf{Valeur métier}} \\ \hline
\multirow{3}{*}{Favoris} & Modifier nom d'un favori                     & L                                            & 5 \\ \cline{2-4} 
                         & Modifier adresse principal du favori         & L                                            & 5                                           \\ \cline{2-4} 
                         & Modifier nom d'un dossier                    & L                                            & 5                                           \\ \hline
\multirow{4}{*}{Mot de passe}           & Proposer génération lors d'une inscription   & L                                            & 4                                          \\ \cline{2-4} 
                                        & Saisie automatique du mot de passe           & XL                                           & 4                                           \\ \cline{2-4} 
                                        & Afficher liste des mots de passes            & M                                            & 4                                           \\ \cline{2-4} 
                                        & Détection des mots de passe pwned            & M                                            & 2                                           \\ \hline
\multirow{3}{*}{Téléchargement}         & Choix du dossier destination                 & M                                            & 4                                           \\ \cline{2-4} 
                                        & Etat du téléchargement                       & XL                                           & 5                                           \\ \cline{2-4} 
                                        & Liste des téléchargement                     & M                                            & 4                                           \\ \hline
\end{tabular}
\end{table}

\newpage


\section{Architecture de l'application}

L'architecture de l'application est très importante afin de s'assurer d'une bonne évolution et maintenabilité. Ainsi, dans cette section nous présentons l'architecture principale de l'application à travers un diagramme de paquetage, un diagramme de classe et un diagramme de séquence. \\
D'autres choix de conception serons présentés dans la section suivante  \textit{\nameref{choix_conception}}

\subsection{Organisation des paquetages}
Les classes et interfaces de notre application ont été structurées dans nos différents paquetages en respectant le modèle MVC. Nous avons également un paquetage \verb|Utils| qui englobe deux autres paquetages \verb|DB| et  \verb|ConfigurationHandle| qui sont respectivement responsables de la gestion de la base de données et de la gestion des fichiers de configuration.

\begin{center}
\begin{tikzpicture}

\begin{class}[text  width=3.5cm]{Controller}{0,-1}
\end{class}	

\begin{class}[text  width=3cm]{Model}{2.5,-2}
\end{class}	

\begin{class}[text  width=3cm]{Vue}{-2.5,-2}
\end{class}	


\begin{class}[text  width=4.5cm]{Utils}{-8,0}
\attribute {}
\attribute {}
\attribute {}
\attribute {}
\attribute {}
\end{class}	


\begin{class}[text  width=3.5cm]{DB}{-8,-0.8}
\end{class}	


\begin{class}[text  width=3.5cm]{ConfigurationHandle}{-8,-1.5}
\end{class}	


\begin{class}[text  width=3.5cm]{APP}{-3,-4}
\end{class}	

\association{Controller}{}{}{Vue}{}{}
\association{Controller}{}{}{Model}{}{}
\association{Utils}{}{}{Controller}{}{}

\draw[->] (APP.south) |- ++(0, -0.5) -- ++(-8,0) --node[above , sloped ,black]{}  +(0 ,1)  |- (Utils.west);
\draw[->] (APP.south) |- ++(0, -0.5) -- ++(8,0) --node[above , sloped ,black]{}  +(0 ,1)  |- (Controller.east);

\end{tikzpicture}
\end{center}

\subsection{Exemple d'un diagramme de classe}
Les deux sections suivantes illustrent le service \textit{Historique} à travers un  diagramme de classe et un diagramme de séquence. Étant donné que, tous les services de l'application (favoris, onglets, cookies, paramètres) suivent la même architecture nous avons trouvé plus pertinent de ne représenter en détail qu'un seul service. Ainsi, ci-dessous vous retrouvez le diagramme de classe du service \textit{Historique}.

\begin{center}
\begin{tikzpicture}

\begin{class}[text  width=7cm]{HistoryController}{0,0}
\attribute {- objet FXML}
\attribute {- historyEntries : ListView<HistoryEntry>}
\operation {+ initialize() : void }
\operation {- searchHistory() : void}
\operation {- getTabsController() : TabsController}
\operation {+ setTabsController(tc : TabsController) : void}
\end{class}	

\begin{class}[text  width=5cm]{HistoryFXML}{-3,-5}
\end{class}	

\begin{class}[text  width=5cm]{HistoryEntry}{3,-5}
\attribute {id : int}
\attribute {link : String}
\attribute {name : String}
\attribute {date : Date}
\operation {getX()}
\operation {setX()}
\end{class}	

\begin{class}[text  width=4cm]{TableHistory}{-9,-1.4}
\end{class}

\association{HistoryController}{}{}{HistoryFXML}{}{}
\association{HistoryController}{}{}{HistoryEntry}{}{}
\association{HistoryController}{}{}{TableHistory}{table}{1}


\end{tikzpicture}
\end{center}

L'implémentation des bases de données pour l'application est présentée plus en détail dans la section \ref{BDD}.

\newpage

\subsection{Exemple d'un diagramme de séquence}
Afin de décrire l'enchainement des évènements entre \verb|Model|, \verb|Vue|, \verb|Controller| et la base de données nous avons réalisé un diagramme de séquence. De même que la section précédente, ce diagramme de séquence est valable pour les favoris, les cookies, les onglets et les paramètres. Ici, nous l'illustrerons avec le service \textit{Historique}.

\begin{center}
\begin{sequencediagram}
\newinst []{User}{:user}
\newthread{App}{:app}
\newinst []{HistFXML}{:histFXML}
\newinst []{HistController}{:histController}
\newinst [1]{TableHistory}{:tableHistory}
\newinst[]{DB}{:DB}

\begin{messcall}{App}{ connection() }{DB}\end{messcall}
\begin{messcall}{User}{ openHistory() }{App}\end{messcall}
\begin{messcall}{App}{ display }{HistFXML}
	\begin{messcall}{HistFXML}{ initialize() }{HistController}
		\begin{messcall}{HistController}{ new TableHistory() }{TableHistory}
			\begin{messcall}{TableHistory}{ create() }{DB}\end{messcall}
			\begin{messcall}{TableHistory}{ load() }{DB}\end{messcall}
		\end{messcall}	
	\end{messcall}
\end{messcall}

\begin{messcall}{User}{ supprime un élément }{HistFXML}
	\begin{call}{HistFXML}{ BTN\_delete.onClick() }{HistController}{delete from interface}
		\begin{messcall}{HistController}{ delete(...) }{TableHistory}
			\begin{messcall}{TableHistory}{ DELETE FROM  }{DB}\end{messcall}
		\end{messcall}
	\end{call}
\end{messcall}


\end{sequencediagram}
\end{center}

\newpage


\section{Choix de conception} \label{choix_conception}
\subsection{Stockage des données}
Lors de la conception de l'application, nous nous sommes demandé s'il serait intéressant d'utiliser une base de données. En effet, nous avons besoin de stocker des informations persistantes telles que l'historique, les favoris, les cookies, etc ... \\
Différentes solutions s'offrent à nous pour réaliser le stockage. Nous pouvions soit utiliser un fichier texte soit utiliser une base de données. \\

\textbf{Les fichiers texte} \\
Ils ont l'avantage d'être simple à créer et à utiliser. De plus si nous utilisons une structure comme JSON ou XML il existe des librairies Java qui permettent de créer et récupérer les informations d'un fichier facilement. \\

\textbf{Une base de données} \\
Est un outils conçu spécialement pour le stockage d'un grand nombre de données. Cependant, il est plus difficile de manipuler les bases de données en Java que les fichiers. De plus, nous devons également stocker les informations enregistrer en base de données sans que l'utilisateur n'est besoin d'installer lui même un serveur de base de données. \\
Cette solution bien que plus compliquée à mettre en place a été retenue car elle présente de nombreux avantages une fois la technologie maitrisée. \\

Ainsi, nous avons choisi une base de données SQLite qui permet d'avoir les informations stockées dans un fichier \verb|.db| représentant la base de données. Ensuite, nous devions faire le lien entre nos classes Java (POJOs) et la table SQL. Une solution serait d'utiliser JavaEE et les annotations mais étant données que nous ne souhaitons pas charger le projet nous avons implémenter différents services qui permettent d'interagir avec la base de données.

\subsubsection{Diagramme de classe} \label{BDD}
Ci-dessous nous représentons l'organisation des classes pour la gestions des tables de la base de données.
La classe \verb|DBConnection| permet de gérer l'accès à la base de données. Et, les implémentations de \verb|Table| font le lien entre les objets Java (POJOs) et les tables de la base de données.

\begin{center}
\begin{tikzpicture}
\begin{interface}[text  width=4.5cm]{Table<T>}{0,0}
\operation {+ create() }
\operation {+ insert(data : T) }
\operation {+ delete(date : T) }
\operation {+ selectAll() }
\operation {+ deleteAll() }
\operation {+ deleteDatas(datas : T[]) }
\operation {+ getDatas() : List<T> }
\end{interface}	


\begin{class}[text  width=5.9cm]{TableHistory}{-3.5,-5.5}
\implement {Table<T>}
\attribute {- datas : List<HistoryEntry>}
\attribute {- DBConnection : DBconnection}
\attribute {- \underline{instance : TableHistory}}
\operation {- TableHistory(co : DBConnection) }
\operation {+ \underline{getInstance(co : DBConnection) : TH}}
\end{class}	

\begin{class}[text  width=5.9cm]{TableCookie}{3.5,-5.5}
\implement {Table<T>}
\attribute {- datas : List<HttpCookie>}
\attribute {- DBConnection : DBconnection}
\attribute {- \underline{instance : TableCookie}}
\operation {- TableCookie(co : DBConnection) }
\operation {+ \underline{getInstance(co : DBConnection) : TC}}
\end{class}	

	
	
\begin{class}[text  width=5.5cm]{DBConnection}{7.5,0}
\attribute {- connection : Connection}
\attribute {- statement : Statement}
\operation {- DBConnection(co : DBConnection) }
\operation {+ \underline{getInstance() : DBConnection}}
\operation {+ connect()}
\operation {+ close()}
\operation {+ createTable(request : String)}
\operation {+ getConnection() : Connection}
\end{class}	
	

\draw[->] (TableHistory.north) |- ++(0, 0.5) -- ++(-3.5,0) --node[above , sloped ,black]{}  +(0 ,-1)  |- (TableHistory.west);

\draw[->] (TableCookie.north) |- ++(0, 0.5) -- ++(-3.5,0) --node[above , sloped ,black]{}  +(0 ,-1)  |- (TableCookie.west);

\draw[->] (DBConnection.north) |- ++(0, 0.5) -- ++(-3.5,0) --node[above , sloped ,black]{}  +(0 ,-1)  |- (DBConnection.west);

\draw[umlcd  style  dashed  line ,->] (TableHistory.south) |- ++(0, -1) |- ++(11,0) --node[above , sloped ,black]{$<<$ use $>>$} (DBConnection.south);

\draw[umlcd  style  dashed  line] (TableCookie.south) -- ++(0, -1.2) ;

\end{tikzpicture}
\end{center}

Un objet de type \verb|DBConnection| est créé au lancement de l'application. Ensuite, pour pouvoir manipuler les tables, nous devons récupérer cet objet pour pouvoir effectuer des traitements dessus, c'est pour cela que nous le passons en paramètre de nos constructeurs \verb|TableHistory| et \verb|TableCookie|. \\

\newpage

\begin{lstlisting}
class HistoryController {

	private TableHistory table;    
    
    @FXML
    intinialize() {
        table = TableHistory.getInstance(DBConnection.getInstance());
        ...
    }
}

class TableHistory implements Table<HistoryEntry> {

	private List<HistoryEntry> historyList;
	private DBConnection DBconnection;

	private static TableHistory instance;
	
	private TableHistory(DBConnection DBconnection) {
		this.DBconnection = DBconnection;
		this.historyList = new ArrayList<HistoryEntry>();
		
		this.create(); this.selectAll();
	}
	
	public static TableHistory getInstance(DBConnection DBconnection) {
		if(instance == null) {
			instance = new TableHistory(DBconnection);
		}
		return instance;
	}
\end{lstlisting}




\newpage

\subsection{Stockage des configurations}
Nous avons souhaité offrir à l'utilisateur la possibilité de configurer certaines options de l'application. 

\subsubsection{Configurer le navigateur}
Premièrement, l'utilisateur peut enregistrer des configurations pour l'intégralité du navigateur web. Par exemple, le moteur de recherche par défaut, s'il souhaite être en mode sombre, etc ... \\
De même que dans la section précédente, nous devons nous poser la question s'il faut utiliser une base de données ou un fichier de configuration. Pour ce cas, étant donné que les configurations possibles ne représentent pas un grand nombreux de données à stocker, il est plus simple de manier un fichier formaté comme JSON.

\begin{verbatim}
config.json
{
    engine: google // le moteur de recherche par défaut est google
}
\end{verbatim}

L'utilisateur à la possibilité de modifier la valeur du moteur de recherche par défaut depuis le menu \textit{paramètre} de l'application. Ce qui aura pour conséquence de modifie la valeur dans le fichier \verb|config.json|.

\subsubsection{Configurer les moteurs recherches}
Ensuite, nous souhaitons que l'utilisateur puissent ajouter des moteurs de recherche ou des raccourcis de recherche au navigateur. Par exemple au lieu d'aller sur Youtube et d'effectuer une rechercher, l'utilisateur peut directement écrire \verb|@youtube nom_video| dans la barre de recherche. \\
Pour ce faire, de même que la configuration du navigateur nous utilisons un fichier JSON en associant un nom et un url.
\begin{verbatim}
webengine.config
{
    youtube: https://youtube.com/result?search_query=
    google: https://google.com/search?q=
}
\end{verbatim}

L'utilisateur a également la possibilité d'ajouter un raccourci \verb|@| depuis les paramètres du navigateur.


\subsubsection{Diagramme de classe}
Nous présentons ci-dessous, un diagramme de classe sur la configuration de l'application

\begin{center}
\begin{tikzpicture}
\begin{interface}[text  width=4.5cm]{ConfigurationFile}{0,-0.6}
\operation {+ load() }
\operation {+ save() }
\end{interface}	

\begin{class}[text  width=4.5cm]{CFNavigator}{-5.5,-4.5}
\implement {ConfigurationFile}
\attribute {- engine : String}
\operation {+ query(url : String) : String}
\operation {+ load() }
\operation {+ save() }
\end{class}	

\begin{class}[text  width=6.5cm]{CFEngineSearch}{0.5,-4.5}
\implement {ConfigurationFile}
\attribute {- availableEngine : HM<String, String>}
\operation {+ addAvailableEngine(engine, url : String) }
\operation {+ isAvaibleEngine(engine : String) : bool. }
\operation {+ load() }
\operation {+ save() }
\end{class}	

\begin{class}[text  width=7cm]{ConfigurationCreator}{-9,0}
\attribute {- configurationFiles : HM<Str, ConfFile>}
\attribute {\underline{- instance : ConfigurationCreator}}
\operation {\underline{+ getInstance() : ConfigurationCreator }}
\operation {+ getConfigurationFile(key : String) : ConfFile }
\operation {- loadConfigurationFiles() }
\end{class}	

\begin{class}[text  width=4cm]{aT}{-10.3,-4.7}
\operation {+ initialize }
\end{class}	

\begin{class}[text  width=4cm]{TabsController}{-10.5,-4.5}
\operation {+ initialize() }
\end{class}	

\aggregation{ConfigurationCreator}{ }{*}{ConfigurationFile}

\draw[->] (ConfigurationCreator.north) |- ++(0, 0.5) -- ++(-4,0) --node[above , sloped ,black]{}  +(0 ,-2)  |- (ConfigurationCreator.west);

\draw[umlcd  style  dashed  line ,->] (TabsController) --node[above ,sloped , black]{} (ConfigurationCreator);
\end{tikzpicture}
\end{center}

C'est dans la méthode \verb|TabsController#initialize()| que nous instancions pour la premièrement fois un objet de type \verb|ConfigurationCreator|. Lors de la création de cet objet, nous exécutons la méthode\verb|loadConfigurationFile| qui va se charger de créer un nouvel objet pour chaque fichier de configuration avec une clé. Ainsi, par la suite nous pourrons récupérer cet objet grâce à la méthode \verb|getConfigurationFile(key)|

\newpage

\begin{lstlisting}
class TabController {
    @FXML
    void initialize() {
	    config = ConfigurationCreator.getInstance();
    }
}

class ConfigurationCreator {
	
	private ConfigurationCreator() {
		configurationsFiles = new HashMap<>();
		loadConfigurationFiles();
	}
	
	public static ConfigurationCreator getInstance() {
	    /* Singleton */
	}

    public void loadConfigurationFiles() {
	    	configurationsFiles.put("configurationFileNavigator", 
		                                      new ConfigurationFileNavigator());
		    configurationsFiles.put("configurationFileEngineSearch", 
		                                      new ConfigurationFileEngineSearch());
	}
}
\end{lstlisting}

Par la suite, les autres contrôleurs pourront récupérer les objets associés aux fichiers de configuration. C'est le cas dans \verb|ParametersController| qui a besoin des fichiers de configuration du navigateur et des moteurs de recherche.

\begin{lstlisting}
class ParametersController {
    @FXML
    private void initialize(){
        // get config from file
        ConfigurationCreator config = ConfigurationCreator.getInstance();

        ConfigurationFileNavigator configNavigator = 
        		(ConfigurationFileNavigator) config.getConfigurationFile("configurationFileNavigator");
        
        ConfigurationFileEngineSearch configEngineSearch =
        		(ConfigurationFileEngineSearch) config.getConfigurationFile("configurationFileEngineSearch");
        		
        /* manipulation des fichiers à travers les deux objets */
}
\end{lstlisting}  

\newpage

\subsection{Complémentarité des controlleurs}
L'une des principales difficultés du projet était de pouvoir communiquer entre controlleurs. En effet, nous avons définie un WebView qui permet d'avoir l'affichage des pages web. Mais lors de l'épic onglet, il a fallu faire en sorte que chaque onglet et une WebView différente. De plus, les boutons de la barre de navigation (suivant, précédent, etc) doivent communiquer seulement avec l'onglet et la WebView courante. \\

Pour résoudre ce problème, nous avons du imbriquer les différents controleurs afin de pouvoir manipuler plusieurs controleurs sur la même vue. Ainsi, la vue définie dans le \verb|main.fxml| avec la fenêtre de navigation et la barre de navigation contient également une imbriquation pour pouvoir utiliser la classe \verb|TabsController|. \\
\verb|<fx:include fx:id="tab" source="tabs.fxml" VBox.vgrow="ALWAYS" />|

\subsubsection{Communication du fils vers le père}
Et dans le controleur du \verb|main.fxml| nous avons
\begin{lstlisting}
@FXML  //  vaut toujours null si manquant
private TabsController tabController;

private void initialize() {
	tabController.setControlsController(this);
	...
}
\end{lstlisting}

La méthode \verb|TabsController.setControlsController| va permettre d'avoir connaissance de la barre de recherche dans la classe \verb|TabsController|. Ainsi, lorsqu'on va changer d'onglet on va pouvoir également changer le nom dans la barre de recherche.
\begin{lstlisting}
/**
 * Assign the window controller instance.
 * @param controlsController the window controller.
 */
public void setControlsController(WebViewController controlsController) {
	this.controlsController = controlsController;
	if (this.controlsController != null) {
		this.addressBar = this.controlsController.getAddressBar();
		this.progressBar = this.controlsController.getProgressBar();
	}
}
\end{lstlisting}

\subsubsection{Communication du père vers le fils}

Nous venons de donner connaissance du controleur parent au controleur fils. Maintenant, nous pouvons utiliser les méthodes définies dans \verb|TabsController|. Par exemple en cliquant sur le bouton \textit{ajouter un onglet} qui est défini dans \verb|main.fxml| on va utiliser la méthode \verb|TabsController#addNewTab()|. 
\begin{lstlisting}
newTabButton.setOnAction(new EventHandler<ActionEvent>() {
	@Override
	public void handle(ActionEvent event) {
		tabController.addNewTab(false);
    	tabController.getCurrentTab().getTab().setText("Loading...");
		NavigationUtils.search(addressBar.getText(), tabController.getCurrentTab().getWebView().getEngine());
	}
});
\end{lstlisting}


\section{Conclusion}
\subsection{Pistes d'amélioration}

Avant de conclure sur ce projet, nous pouvons lister quelques pistes d'amélioration :
\begin{itemize}
\item Pour finaliser le projet, en se basant sur la vélocité, on estime qu'il nous faudrait deux semaines supplémentaires.
\item De même, certain point d'effort ont été mal évalué. Mais de meilleures estimations nécessite une plus grande expérience. 
\item Une meilleure conception dès le début du projet aurait pu nous éviter d'avoir à réaliser des refactors.
\item Suivre jusqu'à la fin du projet le workflow détaillé sur le GitHub car à la fin du projet nous ne nous relisons plus (prise de confiance)
\item Enfin, il manque des tests unitaires car l'architecture est très simple et les tests se baseraient sur l'interface graphique.
\end{itemize}

\subsection{Conclusion sur l'agilité}

L'obligation d'utiliser l'agilité pour ce projet nous a prouvé que ce concept fonctionne bien. En effet, au fur et à mesure des itérations on arrivait à prioriser les tâches et à apporter des modifications sur des points d'effort et des valeurs métier. 
De plus, le fait d'avoir un rendu de l'application chaque semaine est très gratifiant et nous pousse à continuer. \\
Ensuite, la répartition des tâches était équitable. Chacun pouvait travailler sur son service indépendamment puis ensuite le mettre en commun via git. \\
Enfin, nous finissions chaque itération par un sprint review pour se remettre en question et continuer à s'améliorer


\subsection{Conclution du projet}


\end{document}